\documentclass{article}

\usepackage[utf8]{inputenc}
\usepackage[full]{textcomp}
\usepackage{CJKutf8}
\usepackage[lf]{ebgaramond}
% \usepackage[scaled,swashQ]{garamondx}
\usepackage[T1]{fontenc}


\usepackage{enumitem}
<<<<<<< Updated upstream
\usepackage[a4paper,left=.9in, right=.9in, top=1.in, bottom=1.in]{geometry}
=======
\usepackage[a4paper,left=.5in, right=.5in, top=0.75in, bottom=0.25in]{geometry}
>>>>>>> Stashed changes
\usepackage{url}
\usepackage[dvipsnames]{xcolor}
\usepackage{amsmath}

% package settings
\usepackage[
    hidelinks,
    pdfnewwindow=true,
    pdfauthor={Shuhan Zheng},
    pdftitle={Shuhan Zheng Resume (Eng)},
]{hyperref}

\pagestyle{headings}
\markright{\textbf{Shuhan Zheng}}

\setlength\parindent{2em}

\thispagestyle{empty}

% define cv section
\newcommand{\cvsection}[1]{\section*{\rmfamily#1}}
\newcommand{\cvsubsection}[1]{\subsection*{\rmfamily\hspace{1.5em}#1}}

\begin{document}

% Begins the file
\begin{center}
    \Huge{
    \rmfamily
    \textbf{Shuhan Zheng}}
\end{center}
\vspace{15pt}    

\setlength{\parskip}{0.5pt}
\renewcommand{\arraystretch}{1.25}

% Contact Information

\noindent University of Toronto St.George Campus (UTSG)  \hfill Cell number: (+1) 438-495-6697

\noindent Physics and Astronomy Undergraduate \hfill {Email: \texttt{shuhan.zheng@mail.utoronto.ca}}


\setlength{\parskip}{2pt}




% Summary / Statement
%\cvsection{Summary / Statement}
%\indent

%\hangindent=2emI am an undergraduate student studying physics and astronomy & astrophysics at University of Toronto. Currently my 

% Education
\cvsection{Education}
\indent 

\textbf{University of Toronto}, Toronto ON, Canada.

\hspace{2em} HBSc., Major in Physics, University of Toronto, 2020 to 2024 (expc.).

\hspace{2em} HBSc., Major in Astronomy and Astrophysics, University of Toronto, 2020 to 2024 (expc.).

\textbf{Marianopolis College}, Montreal QC, Canada.

\hspace{2em} Pure \& Applied Science DEC, 2018 to 2020.

<<<<<<< Updated upstream
\cvsection{Relevant Experiences}

=======
\cvsection{Research Experiences}


\hspace{2em}In the summer of 2023, I worked as a research intern in Dr. Xin Tong's Polarized Neutron group at China Spallation Neutron Source (CSNS), a research facility operated by China Academy of Sciences, Institute of High Energy Physics. During this time, I collaborated with Dr. Xin Tong's \footnote[1]{Dr. Tong is a \textbf{lead scientist} in the field of polarized neutron science at Oak Ridge National Laboratory and CSNS. \newline See more info about Dr. Tong's Google Scholar page: https://scholar.google.com/citations?user=Iy4X3WAAAAAJ} Polarized Neutron Group. Specifically, I worked with Dr. Tianhao ("Radian") Wang\footnote[2]{Dr. Wang is one of the Ph.D. supervisors and researchers at Oak Ridge National Laboratory and CSNS. \newline See more info at Dr. Wang's ResearchGate page: https://www.researchgate.net/profile/Tianhao-Wang-14} and Dr. Ahmad Salman to develop a polarized neutron imager that can directly observe the magnetic field of a given sample in 3D. I was solely responsible for the design and implementation of various neutron-optical simulations and data analysis programs, including a novel raytracing-based finite element simulation for simulating the detector's output when taking a snapshot of the sample with a diffusive neutron source. Compared to other existing methods, it is faster and can accommodate non-adiabatic transitions and non-coherent neutron beams.

Since September 2021, I have been a member of the Data Processing subsystem of the University of Toronto Aerospace Team Space Systems (UTAT-SS). I was at first in a team that develops, implements, and tests an algorithm that removes smile distortion, then I became the project lead of another team that does the same for keystone distortion. Until UTAT-SS settled on a scientific objective, I was also a member of the Mission Science system, which was scoping for alternative missions for the team's CubeSat project FINCH. The team's design paper of which I am a coauthor has been published in the Smal Satellite Conference 2022\footnote[3]{Miles, A. (n.d.). FINCH: A blueprint for accessible and scientifically valuable remote sensing satellite missions. DigitalCommons@USU. \newline https://digitalcommons.usu.edu/smallsat/2022/all2022/88/}.

% Between December 2020 and January 2021, I led a team of four first-year undergraduate students in a Moon base design contest hosted by Moon Society, which is a scientific advocacy organization for lunar sciences and exploration. The contest requires participant teams to design a lunar base that can support a crew of 30 for 10 years while considering the base's location, construction, operation, and the crew's physical and mental well-being. Despite competing against teams of engineering graduate students and professional engineers, our design paper was selected as a top 10 finalist and was published on the organization's website\footnote[4]{The Moon Society's announcement for contest winners: https://www.moonsociety.org/news/2021/03/10/announcement-of-winners-for-the-moon-societys-first-moon-base-design-contest/}.

\cvsection{Rewards}
\begin{itemize}
    \item Team lead and first author of one of the top 10 finalist submissions in The Moon Society's Moon Base Design Contest\footnote[4]{The Moon Society's announcement for contest winners: https://www.moonsociety.org/news/2021/03/10/announcement-of-winners-for-the-moon-societys-first-moon-base-design-contest/}, 2021 (the smallest, and the only undergraduate team to be selected as a finalist). The goal was to design a lunar base that sustainably supports the survival and well-being of a crew of 30 for 10 years, and then evaluate its cultural and economic significance. The contest welcomed submissions from all levels of education, including engineering graduate students and professionals.
    \item Provincial top 15 rank in Quebec, Canadian Association of Physicists (CAP) Exam, 2019.
\end{itemize}

\cvsection{Leadership Experiences}
>>>>>>> Stashed changes
\begin{itemize}
    \item \textbf{China Spallation Neutron Source (CSNS)}
    \begin{itemize}
        \item Summer research internship in Dr. Xin Tong's Polarized Neutron group, May to July 2023. 
        \begin{itemize}
            \item During this time, I collaborated with Dr. Tianhao (Radian) Wang and Dr. Ahmad Salman to develop a polarized neutron imager that can directly observe the magnetic field of a given sample in 3D.
            \item I was responsible for the design and implementation of various neutron-optical simulations and data analysis programs. One of these programs is a raytracing-based simulation that can simulate the detector's output when taking a snapshot of the sample under a diffusive neutron light source.
        \end{itemize}
    \end{itemize}
    \item \textbf{University of Toronto Physics Student Union (UofT PhySU)}
    \begin{itemize}
        \item Vice-President of Internal and External Affairs, 2023 to 2024 academic year. My role is to organize and coordinate PhySU's collaboration with external partners, as well as with the Department of Physics itself. 
    \end{itemize}
    \item \textbf{University of Toronto Aerospace Team (UTAT) Space Systems}
    \begin{itemize}
        \item Data Processing subsystem
        \begin{itemize}
            \item Since Feburary 2023, Keystone correction project team lead. I am responsible for eawrly feasibility study, progress tracking, and team management.
            \item Since September 2021, memmber of Smile correction project team. I was a part of a team that develops, implements, and tests an algorithm that removes smile distortion.
        \end{itemize}

        \item Mission Science subsystem (disbanded due to mission progress)
        \begin{itemize}
            \item General member, September 2021 to August 2022. Our task was scoping for alternative missions for the team's CubeSat project FINCH, which was published in the team's submission to Small Satellite Conference 2022.
        \end{itemize} 
    \end{itemize}

<<<<<<< Updated upstream
    \item \textbf{Canadian Association of Physicists (CAP) Exam}
    \begin{itemize}
        \item From 2017 through 2019, I attended the CAP prized exam multiple times and ranked higher and higher in the province of Quebec each time. My highest rank was within top 10. 
    \end{itemize}

    \item \textbf{Moon Base Design Contest} by Moon Society, December 2020 to January 2021
    \begin{itemize}
        \item Captain and design lead of a finalist team;
        \item First author of the team's design paper.
    \end{itemize}
=======
    % \item \textbf{Canadian Association of Physicists (CAP) Exam}
    % \begin{itemize}
    %     \item From 2017 through 2019, I attended the CAP prized exam multiple times and ranked higher in the province of Quebec each time. My highest rank was within the top 10. 
    % \end{itemize}
>>>>>>> Stashed changes

    \item \textbf{Marianopolis Propulsion Laboratory}, a student rocketry society at Marianopolis College, Montreal
    \begin{itemize}
        \item Co-executive, 2019 to 2020 academic years;
        \item Outreach director, 2018 to 2020.
    \end{itemize}

    \item \textbf{Various tutoring jobs}, freelance, mid- to long-term, pro bono

\end{itemize}

\cvsection{Notable Skills}

\begin{itemize}
    \item Python programming and LaTeX typesetting;
    
    \item Analog and digital electronics design and prototyping, including applications involving Arduino microcontrollers;
    
    \item Technical drawing, including designing parts with SolidEdge. I am also familiar with making simulations with COMSOL Multiphysics;
    
    \item Scientific and technical writing, including lab reports and scientific journal articles.
    
    
\end{itemize}

\cvsection{Languages}
\hspace*{2em} Native Chinese (Mandarin) \\
\hspace*{2em} Fluent English

\newpage

\cvsection{Course Projects}

\textbf{Computational Physics - Comparison between Fast Mutlipole Method (FMM) and Particle Mesh Method (PM)} \hfill \textit{Fall 2021}

For the course's final project, I did the research for and implemented the FMM and PM algorithms in Python, and compared their performances in terms of accuracy and speed. I also implemented a direct sum method (DSM) for comparison. At the end of the report, I discussed each algorithm's advantages and disadvantages in various situations, such as when the particles are clumped together or when there is a large ensemble of particles. I was rewarded an 85.42\% (A) for this project. 


\noindent\textbf{Electronics Labs - A Serial Photonic Communication Device} \hfill \textit{Winter 2023}

For the course's final project, my project partner and I designed and built a laser-based device that transmits 1s and 0s serially, based on the polarization of each laser pulse. Initially, the design consisted of a laser diode, two polarizers, two photodetectors, a liquid crystal film, a microcontroller, and a beam-splitter. However, due to budgetary and logistics concerns, it was reduced to one polarizer-photodetector pair, a laser diode, a polarizer film attached to a step motor, and a microcontroller. The device was able to transmit with high fidelity, and I was rewarded an 8/10 for this project.

% \cvsection{Academic Performance (Key Courses)}

% \noindent\begin{table}[h]
    % \centering
    \caption{}
    % \begin{tabular}[H]{lllllll}
        % Course Code & Title & Weight & Mark & Grade & Class avg & Semester\\ \hline
        % PHY405 & Electronics Lab & 0.5 & 87 & A & A- & Winter 2023\\ \hline
        % % AST320 & Introduction to Astrophysics & 0.5 & 67 & C+ & C+ & Winter 2023\\ \hline
        % PHY424 & Advanced Physics Lab & 0.5 & 79 & B+ & B- & Fall 2022\\ \hline
        % PHY350 & Electromagnetic Theory & 0.5 &  & CR & C+ & Fall 2022\\ \hline
        % PHY324 & Practical Physics II & 0.5 & 77 & B+ & B+ & Winter 2022\\ \hline
        % PHY407 & Computational Physics & 0.5 & 70 & B- & B  & Fall 2021\\ \hline
        % % PHY256 & Introduction to Quantum Physics & 0.5 & 73 & B & B- & Fall 2021\\ \hline
        % MAT235 & Multivariable Calculus & 1 & 72 & B- & B- & Summer 2021\\ \hline
        % % MAT244 & Ordinary Differential Equations & 0.5 & 57 & D+ & B+ & Fall 2020\\ \hline
    % \end{tabular}
% \end{table}


\end{document}