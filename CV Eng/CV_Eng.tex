\documentclass{article}

\usepackage[utf8]{inputenc}
\usepackage[full]{textcomp}
\usepackage{CJKutf8}
\usepackage[lf]{ebgaramond}
% \usepackage[scaled,swashQ]{garamondx}
\usepackage[T1]{fontenc}

\usepackage{enumitem}
\usepackage[a4paper,left=.5in, right=.5in, top=0.75in, bottom=0.75in]{geometry}
\usepackage{url}
\usepackage[dvipsnames]{xcolor}
\usepackage{amsmath}

% package settings
\usepackage[
    colorlinks=true,
    linkcolor=blue,
    pdfnewwindow=true,
    pdfauthor={Shuhan Zheng},
    pdftitle={Shuhan Zheng Resume (Eng)},
]{hyperref}

\pagestyle{headings}
\markright{\textbf{Shuhan Zheng}}

\setlength\parindent{2em}

\thispagestyle{empty}

% define cv section
\newcommand{\cvsection}[1]{\section*{\rmfamily#1}}
\newcommand{\cvsubsection}[1]{\subsection*{\rmfamily\hspace{1.6em}#1}}

\begin{document}

% Begins the file
\begin{center}
    \Huge{
    \rmfamily
    \textbf{Shuhan Zheng}}
\end{center}
\vspace{20pt}    

\setlength{\parskip}{1pt}
\renewcommand{\arraystretch}{1.25}

% Contact Information

\noindent University of Toronto St.George Campus (UTSG)  \hfill Cell number: (+1) 438-495-6697

\noindent Physics and Astronomy Undergraduate \hfill {Email: \texttt{shuhan.zheng@mail.utoronto.ca}}


\setlength{\parskip}{3pt}




% Summary / Statement
%\cvsection{Summary / Statement}
%\indent

%\hangindent=2emI am an undergraduate student studying physics and astronomy & astrophysics at University of Toronto. Currently my 

% Education
\cvsection{Education}
\indent 

\textbf{University of Toronto}, Toronto ON, Canada.

\hspace{2em} HBSc., Major in Physics, University of Toronto, 2020 to 2024 (expc.).

\hspace{2em} HBSc., Major in Astronomy and Astrophysics, University of Toronto, 2020 to 2024 (expc.).

\textbf{Marianopolis College}, Montreal QC, Canada.

\hspace{2em} Pure \& Applied Science DEC, 2018 to 2020.

\cvsection{Lab Experiences}


\hspace{2em}In summer of 2023, I worked as a research intern in Dr. Xin Tong's Polarized Neutron group at China Spallation Neutron Source (CSNS). During this time, I collaborated with \href{https://www.researchgate.net/profile/Tianhao-Wang-14}{Dr. Tianhao ("Radian") Wang} and \href{URL}{Dr. Ahmad Salman} to develop a polarized neutron imager that can directly observe the magnetic field of a given sample in 3D. I was soley responsible for the design and implementation of various neutron-optical simulations and data analysis programs, including a raytracing-based simulation that can simulate the detector's output when taking a snapshot of the sample under a diffusive neutron light source.

Since September 2021, I have been a member of the Data Processing subsystem of University of Toronto Aerospace Team Space Systems (UTAT-SS). I was first in a team that develops, implements, and tests an algorithm that removes smile distortion, then I became the project lead of another team that does the same for keystone distortion. Until UTAT-SS has settled on a scientific objective, I was also a member of the Mission Science system (disbanded due to mission progress), which was scoping for alternative missions for the team's CubeSat project FINCH. The mission description has beeen published in the team's \href{https://digitalcommons.usu.edu/smallsat/2022/all2022/88/}{submission} to Small Satellite Conference 2022.

Between December 2020 and January 2021, I led a team of four first-year undergraduate students in a Moon base design contest hosted by Moon Society, which is a scientific advocacy organization for lunar sciences and exploration. The contest requires participant teams to design a lunar base that can support a crew of 30 for 10 years, while considering the base's location, construction, operation, and the crew's physical and mental wellbeing. Despite competing against teams of engineering graduate students and professional engineers, our design paper was selected as a finalist and was published on the organization's \href{https://www.moonsociety.org/news/2021/03/10/announcement-of-winners-for-the-moon-societys-first-moon-base-design-contest/}{website}.

\cvsection{Other experiences}
\begin{itemize}
    \item \textbf{University of Toronto Physics Student Union (UofT PhySU)}
    \begin{itemize}
        \item Vice-President of Internal and External Affairs, 2023 to 2024 academic year.
    \end{itemize}

    \item \textbf{Canadian Association of Physicists (CAP) Exam}
    \begin{itemize}
        \item From 2017 through 2019, I attended the CAP prized exam multiple times and ranked higher in the province of Quebec each time. My highest rank was within top 10. 
    \end{itemize}

    \item \textbf{Marianopolis Propulsion Laboratory}, a student rockety society at Marianopolis College, Montreal
    \begin{itemize}
        \item Co-executive, 2019 to 2020 academic years;
        \item Outreach director, 2018 to 2020.
    \end{itemize}

\end{itemize}

\cvsection{Notable Skills}

\hspace{2em}Python programming and LaTeX typesetting, analog and digital electronics design and prototyping, technical drawing, scientific and technical writing.

\end{document}