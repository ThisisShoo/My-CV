% !TEX program = XeLaTeX
\documentclass{article}
\usepackage{zh_CN-Adobefonts_external} 
\usepackage{linespacing_fix}
\usepackage[a4paper, margin=0.75in]{geometry}
\usepackage{cite}
\begin{document}
\pagenumbering{gobble}

\begin{center}
  \Huge{\textbf{郑舒涵}}
\end{center}

\noindent 多伦多大学圣乔治校区 (UTSG) \hfill 学校邮箱:shuhan.zheng@mail.utoronto.ca

\noindent 物理和天文专业本科在读 \hfill 个人邮箱:shuhan\_zheng@qq.com

\setlength{\parskip}{3pt}

\section{教育背景}
\indent

\hspace{2em}\textbf{多伦多大学},加拿大安大略省多伦多市

\hspace{4em}理科荣誉学士学位,物理系,预计2024年毕业;

\hspace{4em}理科荣誉学士学位,天文系,预计2024年毕业。

\hspace{2em}\textbf{Marianopolis College},加拿大魁北克省蒙特利尔市

\hspace{4em}理工专业预科学位,2018年至2020年;

\hspace{4em}社会科学专业预科学位,2017年至2018年。

\hspace{2em}\textbf{圣托马斯高中},加拿大魁北克省蒙特利尔市

\hspace{4em}2016年至2017年,在此就读11年级。

\hspace{2em}\textbf{深圳市福田实验学校},中国广东省深圳市

\hspace{4em}2015年至2016年,在此校国际部中加班就读10年级。

\hspace{2em}\textbf{深圳市南山外国语学校高新中学},中国广东省深圳市

\hspace{4em}2012年至2015年,在此就读7至9年级。

\section{主修课程}
\indent

\hspace{2em}\textbf{数学:}多元微积分、常微分方程、线性代数等;

\hspace{2em}\textbf{物理:}四大力学、电动力学理论、实验物理学、计算物理方法、数字模拟电路等,自学了时序分析;

\hspace{2em}\textbf{天文:}理论天体物理学、天文观测学;

\hspace{2em}\textbf{其他:}计算机编程、科学史、艺术史,自学了天体动力学。

\section{项目经历}
\indent

\begin{itemize}

  \item \textbf{中国散裂中子源}
  \begin{itemize}
    \item 2023年5月至7月,在中国散裂中子源进行暑期实习。在此期间,我的导师是极化中子组的负责人童欣博士。
  \end{itemize}
  \item \textbf{多伦多大学物理系学生会}
  \begin{itemize}
    \item 2023至2024学年成功当选学生会副主席,主要负责学生会内外部事务管理。
  \end{itemize}
  \item \textbf{多伦多大学航空航天队,空间系统分部}
  \begin{itemize}
    \item 数据处理分系统
    \begin{itemize}
      \item 2023年2月至今, Keystone项目组负责人,负责完成项目的前期可行性探讨、路线图规划、项目进度跟踪、人力资源分配等工作,带领项目组成员开发和调试用于纠正高光谱成像中Keystone像差的算法;
      \item 2021年9月至今,Smile项目组成员,负责开发和调试用于纠正Smile像差的算法。
    \end{itemize}
    
    \item 科学目标分系统
    \begin{itemize}
      \item 2021年9月至2022年8月,空间系统部门计划于2024年发射升空FINCH高光谱遥感卫星,我主要负责为其寻找合适的任务目标,包括与负责光学和姿控系统的团队沟通,并最后确定硬件指标。
      %\item 我在空间系统分部的工作成果已发表于2022年的Small Satellite Conference 2022 会议上。参见:https://digitalcommons.usu.edu/smallsat/2022/all2022/88/
      \item 我参与攥写了该团队在Small Satellite Conference 2022 会议上发表的、关于利用高光谱遥感卫星测绘地表农作物及其残留物分布情况的论文。
      \item 论文参见https://digitalcommons.usu.edu/smallsat/2022/all2022/88/
    \end{itemize}
  
  \end{itemize}

  \item \textbf{月面基地设计大赛}
  \begin{itemize}
    \item 2020年12月至2021年1月,总部位于美国加州的科普组织The Moon Society举办了面向全北美所有从业人员及爱好者的月面基地设计大赛,多数参赛团队主要由工程专业研究生,设计成果包括月面基地的选址、太阳与宇宙辐射的防护、环控、能源、生活设施、建造方案、商业运营方案等。经过激烈的竞争,最终成功进入了前十名,并且本团队是其中年龄最小、且全部为本科生的团队。
    \item 我们团队的参赛设计论文已被The Moon Society官方网站收录。参见:https://www.moonsociety.org/
  \end{itemize}

  \item \textbf{Canadian Associaltion of Physicists (CAP) Exam}
  \begin{itemize}
    \item 2017至2019年间,多次参加加拿大物理学家协会(Canadian Associaltion of Physicists)举办的物理竞赛,其中2019年获得了魁北克省前10名的好成绩。
  \end{itemize}

\end{itemize}

\section{其他技能}
\indent

\begin{itemize}
  \item 熟练运用Python进行编程、运用LaTeX进行论文和简历排版;
  \item 熟练运用Arduino编程与电路设计;
  \item 工程制图,包括使用SolidWorks设计零件等;
  \item 英文科技写作,例如实验报告、论文等。
\end{itemize}

\section{兴趣爱好}
\indent
\begin{itemize}
  \item 高一时校运动会参加高二组的1500米跑,获得了第一名;
  \item 在圣托马斯高中时,曾作为主力队员跟随校队参加了区羽毛球赛;
  \item 业余时间爱好自学各类编程技巧;
  \item 热爱阅读,以及骑车、登山等户外运动。
\end{itemize}

\end{document}
